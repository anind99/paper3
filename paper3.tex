% Options for packages loaded elsewhere
\PassOptionsToPackage{unicode}{hyperref}
\PassOptionsToPackage{hyphens}{url}
%
\documentclass[
  11pt,
]{article}
\usepackage{amsmath,amssymb}
\usepackage[]{mathpazo}
\usepackage{ifxetex,ifluatex}
\ifnum 0\ifxetex 1\fi\ifluatex 1\fi=0 % if pdftex
  \usepackage[T1]{fontenc}
  \usepackage[utf8]{inputenc}
  \usepackage{textcomp} % provide euro and other symbols
\else % if luatex or xetex
  \usepackage{unicode-math}
  \defaultfontfeatures{Scale=MatchLowercase}
  \defaultfontfeatures[\rmfamily]{Ligatures=TeX,Scale=1}
\fi
% Use upquote if available, for straight quotes in verbatim environments
\IfFileExists{upquote.sty}{\usepackage{upquote}}{}
\IfFileExists{microtype.sty}{% use microtype if available
  \usepackage[]{microtype}
  \UseMicrotypeSet[protrusion]{basicmath} % disable protrusion for tt fonts
}{}
\makeatletter
\@ifundefined{KOMAClassName}{% if non-KOMA class
  \IfFileExists{parskip.sty}{%
    \usepackage{parskip}
  }{% else
    \setlength{\parindent}{0pt}
    \setlength{\parskip}{6pt plus 2pt minus 1pt}}
}{% if KOMA class
  \KOMAoptions{parskip=half}}
\makeatother
\usepackage{xcolor}
\IfFileExists{xurl.sty}{\usepackage{xurl}}{} % add URL line breaks if available
\IfFileExists{bookmark.sty}{\usepackage{bookmark}}{\usepackage{hyperref}}
\hypersetup{
  pdftitle={The role of quality of life in the non-reciprocal altruistic behaviour},
  pdfkeywords={pandoc, r markdown, knitr},
  hidelinks,
  pdfcreator={LaTeX via pandoc}}
\urlstyle{same} % disable monospaced font for URLs
\usepackage[margin=1in]{geometry}
\usepackage{graphicx}
\makeatletter
\def\maxwidth{\ifdim\Gin@nat@width>\linewidth\linewidth\else\Gin@nat@width\fi}
\def\maxheight{\ifdim\Gin@nat@height>\textheight\textheight\else\Gin@nat@height\fi}
\makeatother
% Scale images if necessary, so that they will not overflow the page
% margins by default, and it is still possible to overwrite the defaults
% using explicit options in \includegraphics[width, height, ...]{}
\setkeys{Gin}{width=\maxwidth,height=\maxheight,keepaspectratio}
% Set default figure placement to htbp
\makeatletter
\def\fps@figure{htbp}
\makeatother
\setlength{\emergencystretch}{3em} % prevent overfull lines
\providecommand{\tightlist}{%
  \setlength{\itemsep}{0pt}\setlength{\parskip}{0pt}}
\setcounter{secnumdepth}{-\maxdimen} % remove section numbering
\usepackage{booktabs}
\usepackage{longtable}
\usepackage{array}
\usepackage{multirow}
\usepackage{wrapfig}
\usepackage{float}
\usepackage{colortbl}
\usepackage{pdflscape}
\usepackage{tabu}
\usepackage{threeparttable}
\usepackage{threeparttablex}
\usepackage[normalem]{ulem}
\usepackage{makecell}
\usepackage{xcolor}
\ifluatex
  \usepackage{selnolig}  % disable illegal ligatures
\fi

\title{The role of quality of life in the non-reciprocal altruistic
behaviour}
\author{true}
\date{03, 22, 2022}

\begin{document}
\maketitle
\begin{abstract}
Altruism is the moral practice of helping others even when it may come
with a personal cost. Although evolution has been a main factor behind
humans developing altruistic behaviours, it does not explain
non-reciprocal altruism (helping strangers). This must be explained
through a combination of evolution and reasoning-based decision making.
In this paper, I study the specific role of quality-of-life in driving
reasoning based altruistic behaviour. To do so, I first select variables
that traditionally capture individual success in life and altruistic
attitudes respectively. Following I study the relationship between these
variables to identify whether there is evidence for personal success
playing a factor in an individual's altruistic attitudes. My finding
suggests that there indeed is evidence for this. Although the identified
variables may not be perfect in capturing the conceptual properties that
we are aiming to capture; it provides an initial validation of the
topic. In the future, more accurate measures that capture altruistic
attitudes such as a game score can be used to further study this topic.
\end{abstract}

\hypertarget{intro}{%
\subsection{Intro}\label{intro}}

Altruism is the moral practice of helping others even when it may come
with a personal cost. Through years of behavioral evolution in social
organisms such as ourselves, altruistic tendencies have emerged as it
served the best interest of the group. Groups of people that took care
of each other we more likely to survive than individualist groups of
people. However, in modern society altruistic behavior doesn't stop at
immediate groups close to the individual.

Today humans act altruistically without chance for repercussion. For
example, people donate blood, help the homeless and even incur costs to
punish those who have harmed others. This cannot be strictly explained
by evolution, through kin selection. In the paper by (Vlerick, 2021),
the have proposed that the co-evolution of gene and culture along with
conscious decision making are the driving forces of non-reciprocal
altruistic behavior.

The former (gene and culture) draws on the idea that evolutionary
process shapes human beliefs. Specifically, competition between groups
select for norms which are most beneficial for survival. As a result,
groups who tend to be more cooperative are more likely to be victorious.
Overtime, these norm selections result in the shaping of genetics. Due
to the punishment of norm violators such as cheaters through various
ways (social criticism, physical punishment), the biological instinct of
being more altruistic may have been selected for. Having said this,
being altruistic in the modern context may sometimes be mal-adaptive.
Incurring personal cost in order to help someone who cannot repay you
back does not seem practical. However, people do so because they
empathize with the experiences of other humans; morally being encouraged
to do so.

This brings us to the latter reason for altruistic behavior; conscious
decision making. (Vlerick, 2021) explained that there is a role that
reasoning players towards the moral decision that result in altruistic
behaviour. Rather than hard-wired motions, intuition-based processes
play a major part. This may even override the automatic, instinctual
responses talked about previously. These processes that underlie moral
decisions such as psychological dispositions of fairness are a unique
property of human altruism. This aspect of altruistic process has yet to
be thoroughly studied closely.

In this paper I will draw on this reasoning-based process to analyse
whether one's individual success plays a role in their likelihood of
acting more altruistically. Drawn from my own intuition that people who
are struggling less in life (whether monetarily, or through another
medium) are more willing to help others. I use the data collected by the
{[}American census survey{]} and select for two groups of variables. The
first representing traditional ideas of individual success, the second
representing one's altruistic tendencies. To follow, I will compare
dependant variables of altruistic responses, grouped by the independent
variables of individual success levels. I will use linear regression
along with t-tests to determine the significance of the relationships
between these variables.

The results suggest that altruistic behaviour does indeed correlate with
individual success. Logically, if a person is financially struggling,
they will have less freedom to help others financially. Similarly, if a
person is not satisfied with their own life, they are less likely to be
in the mental state to help others with theirs. These findings are
important because they provide a deeper insight on what makes humans how
they are. Although individuals may want to perform selfless acts in the
best interest of others in society. Their personal circumstances may not
allow for it.

\hypertarget{data}{%
\subsection{Data}\label{data}}

I will be using the US census survey to conduct this study. Isolating
the research to Americans only reduces the variability in data due to
country of residence. Consequently, I could have chosen any other
country to do this study on as well.

To follow, since I want to study the relationship between Altruism and
Quality of Life, I will be indentifying variables from the data set that
represent either the altruistic characteristic of an individual or their
quality of life.

\hypertarget{variables-that-capture-altruistic-characteristics}{%
\subsubsection{Variables that capture Altruistic
Characteristics}\label{variables-that-capture-altruistic-characteristics}}

\hypertarget{fairv-do-you-think-most-people-would-try-to-take-advantage-of-you-if-they-got-a-chance-or-would-they-try-to-be-fair}{%
\paragraph{FAIRV: Do you think most people would try to take advantage
of you if they got a chance, or would they try to be
fair?}\label{fairv-do-you-think-most-people-would-try-to-take-advantage-of-you-if-they-got-a-chance-or-would-they-try-to-be-fair}}

\hypertarget{helpfulv-would-you-say-that-most-of-the-time-people-try-to-be-helpful-or-that-they-are-mostly-just-looking-out-for-themselves}{%
\paragraph{HELPFULV: Would you say that most of the time people try to
be helpful, or that they are mostly just looking out for
themselves?}\label{helpfulv-would-you-say-that-most-of-the-time-people-try-to-be-helpful-or-that-they-are-mostly-just-looking-out-for-themselves}}

The above two variables indirectly captures an individuals altruistic
intent because of reciprocal altruism. If they believe another person is
likely to be fair/helpful to them, they are more likely to be
fair/helpful towards others.

\hypertarget{helppoor-the-government-should-take-care-of-people-1---strong-agree-5---strong-disagree.}{%
\paragraph{HELPPOOR: The Government should take care of people (1 -
strong agree, 5 - strong
disagree).}\label{helppoor-the-government-should-take-care-of-people-1---strong-agree-5---strong-disagree.}}

\hypertarget{helpnot-the-government-doing-too-much-or-should-do-more-to-help-people}{%
\paragraph{HELPNOT: The Government doing too much or should do more to
help
people}\label{helpnot-the-government-doing-too-much-or-should-do-more-to-help-people}}

\hypertarget{helpsick-the-government-should-help-in-paying-hospital-bills-vs-not-paying-them}{%
\subsubsection{HELPSICK: The Government should help in paying hospital
bills vs not paying
them}\label{helpsick-the-government-should-help-in-paying-hospital-bills-vs-not-paying-them}}

\hypertarget{helpblk-the-government-should-be-helping-imprve-living-standards-of-black-people}{%
\paragraph{HELPBLK: The Government should be helping imprve living
standards of black
people}\label{helpblk-the-government-should-be-helping-imprve-living-standards-of-black-people}}

The above 4 variables are relevant because they represent a person's
willingness to contribute for the greater good of the people. The
government helping people means

\hypertarget{ldctax.-wealth-countries-pay-taxes-to-aid-less-wealthier-ones-5---extremely-agree}{%
\paragraph{LDCTAX. Wealth countries pay taxes to aid less wealthier ones
(5 - extremely
agree)}\label{ldctax.-wealth-countries-pay-taxes-to-aid-less-wealthier-ones-5---extremely-agree}}

\hypertarget{migrpoor-poor-countries-people-should-be-able-to-work-in-wealthier-ones-1---extremely-agree-3---extremely-disagree}{%
\paragraph{MIGRPOOR, Poor countries people should be able to work in
wealthier ones (1 - extremely agree, 3 - extremely
disagree)}\label{migrpoor-poor-countries-people-should-be-able-to-work-in-wealthier-ones-1---extremely-agree-3---extremely-disagree}}

These two variables are relevant for capturing altruistic properties for
the subjects in this dataset specifically because america the wealthiest
country in the world. Aiding people from other countries means people
will be sacrificing their own resources to do so.

\hypertarget{variables-that-capture-an-individuals-quality-of-life-by-societal-definitions}{%
\subsubsection{Variables that capture an individuals quality of life (by
societal
definitions)}\label{variables-that-capture-an-individuals-quality-of-life-by-societal-definitions}}

The following variables represent happiness, health, satisfaction,
financial situation and outlook. Which all are factors that may
contribute to an individuals quality of life (by societal standards).

\hypertarget{coninc-yearly-income-dollars}{%
\paragraph{CONINC: Yearly Income
(DOLLARS)}\label{coninc-yearly-income-dollars}}

\hypertarget{happy-how-happy-are-you-increasing-scale-from-1-to-3}{%
\paragraph{HAPPY: How happy are you? (Increasing scale from 1 to
3)}\label{happy-how-happy-are-you-increasing-scale-from-1-to-3}}

\hypertarget{life-in-general-do-you-find-life-exciting-pretty-routine-or-dull-1---exiting-3---dull}{%
\paragraph{LIFE: In general, do you find life exciting, pretty routine,
or dull? \{1 - exiting, 3 -
dull\}}\label{life-in-general-do-you-find-life-exciting-pretty-routine-or-dull-1---exiting-3---dull}}

\hypertarget{satfin-are-you-satisfied-with-your-financial-situation-1---well-satisfied-3---notsatisfied-at-all}{%
\paragraph{SATFIN: Are you satisfied with your financial situation? (1 -
well satisfied, 3 - Notsatisfied at
all)}\label{satfin-are-you-satisfied-with-your-financial-situation-1---well-satisfied-3---notsatisfied-at-all}}

\hypertarget{finrela-financial-situation-relative-to-others-1-far-below-average-4-far-above-average}{%
\paragraph{FINRELA: Financial situation relative to others (1: far below
average, 4: far above
average)}\label{finrela-financial-situation-relative-to-others-1-far-below-average-4-far-above-average}}

\hypertarget{goodlife-do-you-have-a-good-chance-of-living-a-good-life-given-the-circumstances-in-america-1---agree-2---disagree}{%
\paragraph{GOODLIFE: Do you have a good chance of living a good life
given the circumstances in America? (1 - agree, 2 -
disagree)}\label{goodlife-do-you-have-a-good-chance-of-living-a-good-life-given-the-circumstances-in-america-1---agree-2---disagree}}

\hypertarget{hlthphys-rating-of-physical-health-increasing-better}{%
\paragraph{HLTHPHYS: Rating of physical health (Increasing =
Better)}\label{hlthphys-rating-of-physical-health-increasing-better}}

\hypertarget{hlthmntl-rating-of-mental-health-increasing-better}{%
\paragraph{HLTHMNTL: Rating of mental health (Increasing =
Better)}\label{hlthmntl-rating-of-mental-health-increasing-better}}

\hypertarget{satsoc-are-you-satisfied-with-your-social-activities-and-relationships-1-not-satisfied-at-all-5-very-satisfied}{%
\paragraph{SATSOC: Are you satisfied with your social activities and
relationships? (1: not satisfied at all, 5: very
satisfied)}\label{satsoc-are-you-satisfied-with-your-social-activities-and-relationships-1-not-satisfied-at-all-5-very-satisfied}}

To start exploring these variables, it may be beneficial to look at
their mean and distributions.The tables below represent this
information.

\begin{table}

\caption{\label{tab:unnamed-chunk-3}Table I: Means and standard Deviations of Selected Variables}
\centering
\begin{tabular}[t]{l|l|l}
\hline
Attribute\_Name & Mean & Standard\_Deviation\\
\hline
helppoor & 2.68249145461451 & 1.25425762080732\\
\hline
helpnot & 2.85856650057493 & 1.29309323801711\\
\hline
helpsick & 2.37571374191092 & 1.26740292785829\\
\hline
helpblk & 3.02690238278247 & 1.45734122894923\\
\hline
ldctax & 3.17711962833914 & 1.16180211089949\\
\hline
migrpoor & 2.3760045924225 & 0.944379970052548\\
\hline
fairv & 2.1812865497076 & 0.747392213822805\\
\hline
helpfulv & 2.0979381443299 & 0.86612079823994\\
\hline
coninc & 55956.2264089826 & 47369.6878711859\\
\hline
happy & 2.03487792725461 & 0.651077255301461\\
\hline
life & 1.69052079430498 & 0.561899413205503\\
\hline
satfin & 1.92729083665339 & 0.739352637357047\\
\hline
finrela & 2.93830845771144 & 0.96191478009661\\
\hline
goodlife & 2.74099099099099 & 1.07322652071708\\
\hline
hlthphys & 2.67382920110193 & 1.03770389194012\\
\hline
hlthmntl & 2.51855925213088 & 1.01555520732269\\
\hline
satsoc & 2.78188845941469 & 1.03430416701141\\
\hline
\end{tabular}
\end{table}

Since most of these variables are categorical, just looking at mean and
standard deviation does not give us an idea of it's distribution. Below
I present their distribution as a frequency (in a table).

\begin{table}

\caption{\label{tab:unnamed-chunk-4}Table II: Categorical Variable Frequencies}
\centering
\begin{tabular}[t]{l|l|l|l|l|l}
\hline
AttributeName & 1 & 2 & 3 & 4 & 5\\
\hline
helppoor & 0.25636 & 0.12723 & 0.39309 & 0.12419 & 0.099126\\
\hline
helpnot & 0.20659 & 0.14795 & 0.37447 & 0.12227 & 0.14872\\
\hline
helpsick & 0.34907 & 0.17929 & 0.30187 & 0.08641 & 0.083365\\
\hline
helpblk & 0.21522 & 0.15642 & 0.26095 & 0.12106 & 0.24635\\
\hline
ldctax & 0.087108 & 0.19454 & 0.31823 & 0.25436 & 0.14576\\
\hline
migrpoor & 0.16073 & 0.44202 & 0.28588 & 0.083238 & 0.028129\\
\hline
fairv & 0.20468 & 0.40936 & 0.38596 &  & \\
\hline
helpfulv & 0.33063 & 0.2408 & 0.42857 &  & \\
\hline
 &  &  &  &  & \\
\hline
happy & 0.19507 & 0.57499 & 0.22995 &  & \\
\hline
life & 0.36043 & 0.58861 & 0.050955 &  & \\
\hline
satfin & 0.31225 & 0.44821 & 0.23954 &  & \\
\hline
finrela & 0.071393 & 0.24328 & 0.39876 & 0.24876 & 0.037811\\
\hline
goodlife & 0.10886 & 0.35435 & 0.28153 & 0.19745 & 0.057808\\
\hline
hlthphys & 0.1449 & 0.27879 & 0.37493 & 0.16033 & 0.041047\\
\hline
hlthmntl & 0.1592 & 0.36101 & 0.31537 & 0.13088 & 0.033544\\
\hline
satsoc & 0.099945 & 0.3106 & 0.35422 & 0.17808 & 0.057151\\
\hline
\end{tabular}
\end{table}

All the above categorical variables are on a 5-point scale. Therefore a
single table is suitable for visualising their frequency counts. An
interesting finding is that people usually are hesitant to answer at
extreme values. Most of the time the middle value is the answer with the
highest frequency.

The following variables are significantly right skewed: helppoor,
migrpoor, helpnot, helpsick, life. This tells us that most people agree
that the government should be helping people (specifically the poor and
sick). The life (satisfaction) variable being skewed to the right tells
us that more people are dissatisfied with their life than the contrary.
The only variable that is significantly left skewed is ldctax. This
implies that while people are willing for the government to be helping
people in their own country (including themselves). They are much less
willing to incur costs to help people they aren't directly involved
with.

To follow, ``Coninc: The income of the individual in dollars'' is the
only continuous variable, it is suitable to represent using a histogram.
Below I present it's distrbution along with a normal overlay curve.

\includegraphics{paper3_files/figure-latex/unnamed-chunk-5-1.pdf}

\begin{verbatim}
## geom_function: na.rm = FALSE
## stat_function: fun = function (x, mean = 0, sd = 1, log = FALSE) 
## .Call(C_dnorm, x, mean, sd, log), n = 101, args = list(mean = 55956.2264089826, sd = 47369.6878711859), na.rm = FALSE, xlim = NULL
## position_identity
\end{verbatim}

We can see that the yearly income of individuals (barring the positive
outlier), is fairly normally distrbuted as we would anticipate.

To keep the inital exploration of the relationships between concise, and
reducing the possible correlations between quality of life attributes I
will only be selecting specific variables to presenting the distirbution
counts of.

For the dependent variables representing altruism I will be using
``helppoor,'' ``helpsick,'' ``migrpoor,'' ``ldctax.''

For the independent variables representing quality of life I will be
using ``satfin,'' ``happy,'' ``hlthphys'' and ``inc\_group.'' This last
variable I have created; which categorises the income level of the
individual within a bracket. To visualize these relationships, a bar
chart is suitable. Each of these variables are less likely to be
co-dependent than for example ``mental health'' and ``happiness,'' and
therefore can have more weight when representing the concept of
``quality of life.''

Below are tables of the mean altruistic scores, grouped by the
independent ``quality of life'' variables.

\begin{table}

\caption{\label{tab:unnamed-chunk-6}Table III: Mean Altruistic Characteristic Scores by Happiness and Physical Health Level}
\centering
\begin{tabular}[t]{r|r|r|r|r|r}
\hline
happy & hlthphys & Migrpoor\_Score\_Mean & ldctax\_Score\_Mean & Helpsick\_Score\_Mean & Helppoor\_Score\_Mean\\
\hline
1 & 1 & 2.521739 & 3.369565 & 2.663044 & 2.771739\\
\hline
1 & 2 & 2.402597 & 3.402597 & 2.792208 & 3.259740\\
\hline
1 & 3 & 2.358696 & 3.173913 & 2.565217 & 2.793478\\
\hline
1 & 4 & 2.428571 & 2.571429 & 1.928571 & 2.000000\\
\hline
1 & 5 & 3.000000 & 3.750000 & 3.250000 & 3.500000\\
\hline
2 & 1 & 2.283186 & 3.150443 & 2.088496 & 2.522124\\
\hline
2 & 2 & 2.292683 & 3.186992 & 2.284553 & 2.593496\\
\hline
2 & 3 & 2.429022 & 3.201893 & 2.223975 & 2.649842\\
\hline
2 & 4 & 2.364341 & 3.023256 & 2.046512 & 2.472868\\
\hline
2 & 5 & 2.750000 & 2.750000 & 2.450000 & 2.850000\\
\hline
3 & 1 & 2.358974 & 3.230769 & 2.384615 & 2.794872\\
\hline
3 & 2 & 2.328358 & 3.149254 & 2.313433 & 2.850746\\
\hline
3 & 3 & 2.409091 & 3.155844 & 2.311688 & 2.668831\\
\hline
3 & 4 & 2.204082 & 2.908163 & 2.020408 & 2.346939\\
\hline
3 & 5 & 2.312500 & 3.187500 & 2.218750 & 2.625000\\
\hline
\end{tabular}
\end{table}

\begin{table}

\caption{\label{tab:unnamed-chunk-6}Table IV: Mean Altruistic Characteristic Scores by Income Group and Financial Satisfaction}
\centering
\begin{tabular}[t]{r|l|r|r|r|r}
\hline
satfin & inc\_group & Migrpoor\_Score\_Mean & ldctax\_Score\_Mean & Helpsick\_Score\_Mean & Helppoor\_Score\_Mean\\
\hline
1 & 0-20k & 2.457143 & 3.085714 & 2.028571 & 2.457143\\
\hline
1 & 20k-40k & 2.445946 & 3.418919 & 2.540541 & 3.067568\\
\hline
1 & 40k-80k & 2.478261 & 3.161491 & 2.478261 & 2.739130\\
\hline
1 & 80k-100k & 2.256757 & 3.405405 & 2.554054 & 2.837838\\
\hline
1 & 100k-200k & 2.297101 & 3.217391 & 2.217391 & 2.630435\\
\hline
2 & 0-20k & 2.333333 & 2.888889 & 2.104575 & 2.431373\\
\hline
2 & 20k-40k & 2.366667 & 3.172222 & 2.288889 & 2.666667\\
\hline
2 & 40k-80k & 2.359606 & 3.285714 & 2.330049 & 2.817734\\
\hline
2 & 80k-100k & 2.258065 & 3.209677 & 2.629032 & 2.741935\\
\hline
2 & 100k-200k & 2.296875 & 3.531250 & 2.359375 & 3.031250\\
\hline
3 & 0-20k & 2.383648 & 2.792453 & 2.119497 & 2.421384\\
\hline
3 & 20k-40k & 2.398058 & 3.097087 & 2.067961 & 2.378641\\
\hline
3 & 40k-80k & 2.546875 & 3.234375 & 2.328125 & 2.468750\\
\hline
3 & 80k-100k & 2.250000 & 3.500000 & 2.083333 & 2.833333\\
\hline
3 & 100k-200k & 2.166667 & 3.500000 & 2.250000 & 2.666667\\
\hline
\end{tabular}
\end{table}

There are two interesting findings from the tables above. First, the
willingness of individuals agreeing that the government should be
helping sick people increases as their physical health score declines.
This may be because of personal gain, but also because their
empathy/understanding that people in a similar position. Similarly, the
willingness of people agreeing that the government should be helping
poor people decreases with income group.

Below are counts of altruistic responses, in relation to the identified
``quality of life'' variables. I visualize them using a box plot and
facet on an additional independent variable.

\includegraphics{paper3_files/figure-latex/unnamed-chunk-7-1.pdf}
\includegraphics{paper3_files/figure-latex/unnamed-chunk-7-2.pdf}
\includegraphics{paper3_files/figure-latex/unnamed-chunk-7-3.pdf}
\includegraphics{paper3_files/figure-latex/unnamed-chunk-7-4.pdf}
\includegraphics{paper3_files/figure-latex/unnamed-chunk-7-5.pdf}
\includegraphics{paper3_files/figure-latex/unnamed-chunk-7-6.pdf}
\includegraphics{paper3_files/figure-latex/unnamed-chunk-7-7.pdf}
\includegraphics{paper3_files/figure-latex/unnamed-chunk-7-8.pdf}

\hypertarget{helppoor-by-income-group-and-financial-satisfaction}{%
\paragraph{Helppoor by Income Group and Financial
Satisfaction}\label{helppoor-by-income-group-and-financial-satisfaction}}

Lower income groups are right skewed; meaning people with lower incomes
are more willing for the government to give financial aid to poor
people, as expected. To add, higher financial (lower number)
satisfaction also results in a right skew. Wealthier people who are
satisfied are more likely to take an altruistic stance.

\hypertarget{helppoor-by-happiness-and-physical-health}{%
\paragraph{Helppoor by Happiness and Physical
Health}\label{helppoor-by-happiness-and-physical-health}}

Lower physical health levels are right skewed; people who are in worse
physical condition are more likely to accept government financial aid
towards poor people. This may be due to financial reasons (expensive
health care in america), but also could be because they are empathetic
towards poorer people due to their own circumstance.

\hypertarget{helpsick-by-income-group-and-financial-satisfaction}{%
\paragraph{Helpsick by Income Group and Financial
Satisfaction}\label{helpsick-by-income-group-and-financial-satisfaction}}

As previously found Lower income groups are right skewed. To add, high
financial satisfaction in addition to a higher income group also results
in a right skew. Wealthier people who are satisfied are more likely to
take an altruistic stance.

\hypertarget{helpsick-by-happiness-and-physical-health}{%
\paragraph{Helpsick by Happiness and Physical
Health}\label{helpsick-by-happiness-and-physical-health}}

Almost all distributions are right skewed.

\hypertarget{migrpoor-by-income-group-and-financial-satisfaction-or-happiness-and-physical-health}{%
\paragraph{Migrpoor by (Income Group and Financial Satisfaction) or
(Happiness and Physical
Health)}\label{migrpoor-by-income-group-and-financial-satisfaction-or-happiness-and-physical-health}}

Almost all distributions are right skewed. That means most people agree
that poor people should be able to move to other countries for better
opportunity

\hypertarget{ldctax-by-income-group-and-financial-satisfaction}{%
\paragraph{Ldctax by Income Group and Financial
Satisfaction}\label{ldctax-by-income-group-and-financial-satisfaction}}

Higher Income Groups are left skewed. Meaning people who are wealthy are
less willing to take an altruistic stance that may not be reciprocal at
all (helping people in other countries). This may be because they cannot
relate to their situation.

\hypertarget{ldctax-by-by-happiness-and-physical-health}{%
\paragraph{Ldctax by by Happiness and Physical
Health}\label{ldctax-by-by-happiness-and-physical-health}}

Most groups are left skewed, meaning people are less willing to help
others in a different country without expecting something in return.
However, an interesting finding is that people who are in high physical
health along with level of happiness have a right skewed distrbution.
This implies that happier and healthier people are more likely to take
an altruistic stance.

To delve into these relationships more deeply, I will now be examining
them quantitatively using statistically models.

\hypertarget{results}{%
\subsubsection{Results}\label{results}}

To start I will be analyzing the relationship between the categorical
variables representing quality of life, with the categorical variables
representing altruistic attitudes. An appropriate analysis to determine
the dependence between two categorical variables is the chi squared
test. Below I provided a table of the chi-squared p value of between
each of the altruistic variables and the quality of life variables. I
will be using a significance level of 5\%.

\begin{table}

\caption{\label{tab:unnamed-chunk-8}Table IV: P Value of Chi Square Test Comparing Altruistic and Quality of Life Categorical Variables}
\centering
\begin{tabular}[t]{l|l|r|r|l|l|l|r}
\hline
Quality of Life Attribute & helppoor & helpnot & helpsick & helpblk & ldctax & migrpoor & helpfulv\\
\hline
happy & 1.1e-07 & 0.0e+00 & 0.0e+00 & 0.002 & > 0.05 & > 0.05 & 1.0e-03\\
\hline
life & > 0.05 & 9.6e-04 & 4.9e-03 & > 0.05 & 0.016 & > 0.05 & NA\\
\hline
satfin & 3.3e-17 & 0.0e+00 & 0.0e+00 & 1.9e-06 & 2.2e-06 & > 0.05 & 2.0e-07\\
\hline
finrela & 2e-14 & 0.0e+00 & 0.0e+00 & 1.2e-08 & 0.001 & 0.025 & 1.6e-02\\
\hline
goodlife & 5.2e-11 & 0.0e+00 & 0.0e+00 & 1.7e-09 & 1.8e-09 & 2.4e-08 & 6.5e-03\\
\hline
hlthphys & 0.0025 & 3.4e-03 & 8.0e-04 & 0.0025 & 0.00047 & > 0.05 & 6.1e-03\\
\hline
hlthmntl & 1.8e-09 & 1.0e-07 & 0.0e+00 & 0.0026 & 8.8e-09 & 5.9e-06 & 2.2e-04\\
\hline
satsoc & 5.5e-07 & 5.0e-05 & 6.0e-07 & 0.019 & 0.00043 & 0.0067 & 8.2e-03\\
\hline
\end{tabular}
\end{table}

As seen from this chart:

\begin{verbatim}
* Happiness has a significant relationship with all factors besides ldctax and migrpoor. 
* life (satisfaction) has a signifiant relationship with helpnot, helpsick, and ldctax
* Financial Satisfaction (satfin) and Physical health (hlthphys) has a signficant relatioship with all attributes besides migrpoor
* finrela (Relational income), goodlife, and satsoc has a signifianct relationship with all altruistic attributes
\end{verbatim}

As expected, the variables representing quality of life and the
variables represent altruistic characteristics are mostly correlated.
However, the extent of these are yet to be determined.

To follow and supplement this analysis I will use ordinal logistic
regression to model each of the relationships between the altruistic
variables and the quality of life attributes.

Here is formula for my ordinal logistic regression model:

l o g i t ( P ( Y ≤ j ) ) = B\_\{j0\} + B\_\{j1\} x\_1 ; where l o g i t
( P ( Y ≤ j ) ) = log frac\{P ( Y ≤ j )\}/\{P ( Y \textgreater{} j )\}

Below is a table that contains the beta coefficient of these models
comparing each of the altruism variables with the quality of life
attributes. The models where the relationship was not significant
(according to chi-squared) contains a ``p \textgreater{} 0.05'' marker.

\begin{table}

\caption{\label{tab:unnamed-chunk-9}Table V: Coefficient of Logistic Regression of Altruistic and Quality of Life Variables}
\centering
\begin{tabular}[t]{l|l|r|r|l|l|l|r}
\hline
Quality of Life Attribute & helppoor & helpnot & helpsick & helpblk & ldctax & migrpoor & helpfulv\\
\hline
happy & -0.145 & -0.1740 & -0.1830 & -0.0574 & P > 0.05 & P > 0.05 & 0.0950\\
\hline
life & P > 0.05 & -0.0557 & -0.0756 & P > 0.05 & -0.04 & P > 0.05 & NA\\
\hline
satfin & -0.213 & -0.1640 & -0.1610 & -0.00778 & -0.124 & P > 0.05 & 0.2390\\
\hline
finrela & 0.111 & 0.0621 & 0.0643 & -0.0761 & 0.119 & -0.127 & -0.0999\\
\hline
goodlife & -0.0988 & -0.0845 & -0.0904 & 0.0233 & -0.0522 & 0.209 & 0.0768\\
\hline
hlthphys & -0.1 & -0.0629 & -0.0878 & 0.0663 & -0.1 & P > 0.05 & 0.1240\\
\hline
hlthmntl & -0.204 & -0.1450 & -0.1990 & -0.0904 & -0.184 & -0.0946 & 0.1870\\
\hline
satsoc & -0.173 & -0.1580 & -0.1810 & -0.0633 & -0.0522 & 0.0381 & 0.2060\\
\hline
\end{tabular}
\end{table}

It may be difficult to realize what these numbers represent. Here is an
interpretation of coefficients provided:

Let B be the logistic regression coeffcient of model where Y represents
an altruistic variable and X represents a quality of life attribute.
With 1 point increase in X, logit of Y (log of the odds of Y = y as
opposed to another value) increases by B amount. In terms of odds, the
odds of Y being a specific number is multiplied by exp(B) when X
increases by 1.

Looking at the values in the above table realize that the correlation
between the variables are actually quite moderate; with all the
coefficients having an absolute value less than 0.25. Interpreting the
relationship between helppoor and happy specifically; the logit of
helppoor decreases by 0.145 points when the happiness score of the
individual increases by 1. Exp(0.145) is 0.865. Therefore in other
words, the odds of the individual disagreeing that the government should
be helping poor people decreases by 13.5\% when their happiness score
increases by 1.

As seen above the quality-of-life attributes of hlthmntl, satfin and
satsoc have a relatively high effect on altruistic attitudes.Therefore,
below I will provide a visualization of the effects on altruistic
variables specifically by the quality of life attributes of hlthmntl,
satsoc and satfin respectively.

\includegraphics{paper3_files/figure-latex/unnamed-chunk-10-1.pdf}
\includegraphics{paper3_files/figure-latex/unnamed-chunk-10-2.pdf}
\includegraphics{paper3_files/figure-latex/unnamed-chunk-10-3.pdf}
\includegraphics{paper3_files/figure-latex/unnamed-chunk-10-4.pdf}
\includegraphics{paper3_files/figure-latex/unnamed-chunk-10-5.pdf}
\includegraphics{paper3_files/figure-latex/unnamed-chunk-10-6.pdf}
\includegraphics{paper3_files/figure-latex/unnamed-chunk-10-7.pdf} As
seen from these charts, here are some interesting findings: * With the
increase of mental health the probability that an individual highly
agrees (Y = 1) that the government should be helping people (helppoor,
helpnot, helpsick, helpblk) increases significantly. * With the increase
of mental health the probability that individual highly disagrees (Y =
5) that the government should be helping black people decreases
significantly. * With the increase of mental health the probability that
individuals highly agree (Y = 3) that most people will be helpful
towards them significantly increases, while the probability that they
highly disagree (Y = 1) signifantly decreases.

\includegraphics{paper3_files/figure-latex/unnamed-chunk-11-1.pdf}
\includegraphics{paper3_files/figure-latex/unnamed-chunk-11-2.pdf}
\includegraphics{paper3_files/figure-latex/unnamed-chunk-11-3.pdf}
\includegraphics{paper3_files/figure-latex/unnamed-chunk-11-4.pdf}
\includegraphics{paper3_files/figure-latex/unnamed-chunk-11-5.pdf}
\includegraphics{paper3_files/figure-latex/unnamed-chunk-11-6.pdf}
\includegraphics{paper3_files/figure-latex/unnamed-chunk-11-7.pdf} As
seen from these charts, here are some interesting findings: * With the
increase of financial satisfaction the probability that an individual
highly agrees (Y = 1) that the government should be helping people
(helppoor, helpnot, helpsick) increases significantly. * With the
increase of financial satisfaction the probability that an individual
agrees (y = 2) that the people in richer countries should be paying tax
to poorer countries in order to help people (altruism) increases
significantly. * With the increase of financial satisfaction, the
probability the probability that inidividual highly agrees (y = 3) other
people will be helpful towards them increases significantly, and the
probability that they highly disagree (Y = 1) decreases significantly.

\includegraphics{paper3_files/figure-latex/unnamed-chunk-12-1.pdf}
\includegraphics{paper3_files/figure-latex/unnamed-chunk-12-2.pdf}
\includegraphics{paper3_files/figure-latex/unnamed-chunk-12-3.pdf}
\includegraphics{paper3_files/figure-latex/unnamed-chunk-12-4.pdf}
\includegraphics{paper3_files/figure-latex/unnamed-chunk-12-5.pdf}
\includegraphics{paper3_files/figure-latex/unnamed-chunk-12-6.pdf}
\includegraphics{paper3_files/figure-latex/unnamed-chunk-12-7.pdf}

As seen from these charts, here are some interesting findings: * With
the increase of social satisfaction the probability that an individual
highly agrees (Y = 1) that the government should be helping people
(helppoor, helpnot, helpsick, helpblk) increases significantly. * With
the increase of social satisfaction the probability that individual
highly disagrees (Y = 5) that the government should be helping black
people decreases significantly. * With the increase of social
satisfaction, the probability the probabilty that individual highly
agrees (y = 3) other people will be helpful towards them increases
significantly, and the probability that they highly disagree (Y = 1)
decreases significantly.

While we have seen the relationships between the categorical variables
capturing quality of life and altruistic properties, we have yet to
study the impact of the continuous variable representing yearly income.

I will be using a linear regression model to study these relationships.
Below is a table containing the coefficients of linear regression
individually between Coninc and each Altruistic Attribute.

Here is the formula for relevant for this: Y = B\_0 + B\_1X\_1. The
table contains coefficients of X.

\begin{table}

\caption{\label{tab:unnamed-chunk-13}Table VI: Coefficient of Linear Regression of Altruistic Variables and Yearly Income}
\centering
\begin{tabular}[t]{r|r|r|r|r|l|l}
\hline
helppoor & helpnot & helpsick & helpblk & ldctax & migrpoor & helpfulv\\
\hline
3691.152 & 1759.942 & 1927.784 & -1758.809 & 4699.85 & P > 0.05 & P > 0.05\\
\hline
\end{tabular}
\end{table}

There are several interesting findings from the linear regression models
above:

\begin{verbatim}
* Migrpoor and Helpfulv do not have a significant relationship with Coninc
* With increase of income by the following decrease:
  + The amount an individual agrees that the government should be helping poor, sick and all people (helppoor/helpsick/helpnot)
  + The amount an individual agrees that rich people should be paying tax to poorer countries (ldctax)
* With the increase of income, the amount an individual agrees that the government should be helping black people financially (helpblk) increases.
\end{verbatim}

While the above is a linear regression model of each altruistic factor
individually, I will now provide a model containg all altruistic
factors. The formula being: Coninc = B\_0 + sum\_i(B\_iX\_i). Below is
table containing the coefficients of this model.

\begin{table}

\caption{\label{tab:unnamed-chunk-14}Table VII: Coefficient of Multiple Linear Regression of Yearly Income and Altruistic Variables}
\centering
\begin{tabular}[t]{l|l|l|r|r|r|l}
\hline
helppoor & helpnot & helpsick & helpblk & ldctax & migrpoor & helpfulv\\
\hline
P > 0.05 & P > 0.05 & P > 0.05 & -5358.162 & 7684.125 & -4397.907 & P > 0.05\\
\hline
\end{tabular}
\end{table}

When all altruistic attributes are accounted for together, here are
interesting findings:

\begin{verbatim}
* helppoor, helpnot, helpsick and helpfulv do not have a significant relatioship with Coninc
* With the increase of income, the following decreases:
  + The amount an individual agrees that rich people should be paying tax to poorer countries (ldctax)
  + With the increase of income, the amount an individual agrees that poor people in some countries should be able to move to richer countires for better opportunite (migrpoor) decreases.
* With the increase of income, the amount that an individual agrees that the government should be helping black people financially (helpblk) specifically, increases.
\end{verbatim}

\hypertarget{discussion}{%
\subsubsection{Discussion}\label{discussion}}

Altruism is an aspect of human intelligence that has evolved naturally.
It has helped groups survived better in a harsh natural environment
where cooperation is necessary. The most common type of altruism is
reciprocal. To elaborate, people may be altruistic towards their family
and friends because they expect altruistic attitudes back. Through
intention, this type of altruism is logical. People who were selectively
altruistic towards their relatives were probably more likely to pass on
their genes. Thus, preserving the characteristic in the presence of
natural selection.

On the contrary, it is unclear how non-reciprocal altruism arose. In the
modern-day people feel a moral pressure to act altruistically towards
others. Part of it may be due to preserving their social image, but
there are many examples where this is not the case. To specify, people
donate money, endanger themselves to protect others, help the homeless,
etc. This latter type of altruism cannot be easily explained through
genetic theory. There are ongoing discussions.

(Vlerick, 2021) have proposed that the combination of gene and culture
is the cause of non-reciprocal altruism. They explain that due to the
benefits of altruistic traits social norms have been shaped to reward
such actions. Reversely, these social norms/rewards have resulted in the
selection of altruistic traits in general. Thus, through a circular
effect, non-reciprocal altruism has become a prevalent trait in humans.
To supplement, (Vlerick, 2021) also adds that conscious decision making
also plays a part in an individual's altruistic attitude. The complex
behaviour of the human mind cannot be easily explained through genetics.
Their psychological disposition, life experience and moral compass play
a significant part.

Consequently, my goal in this paper was to explore this conscious aspect
of non-reciprocal altruism. In specific, I wanted to study whether the
quality of an individual's life (by societal definitions) influences the
extent of altruistic attitude towards others (non-reciprocal). I
hypothesize that the better the quality of an individual's life, the
more inclined and able they are to take on altruistic attitudes. To
study this, I used data collected by the American social survey and
selected questions which I decided were either related to altruistic
attitude or quality of life.

I began the analysis by looking at the mean/count of the altruistic
responses grouped by quality-of-life attributes (of select variables). I
found that the mean scores ``helppoor'' and ``helpsick'' decreased with
``income group'' and ``physical health'' respectively. This is aligned
with my hypothesis. However, rather than altruistic, these findings are
mostly self-serving. Individuals who have worse health would benefit
more from government help. Similarly, lower wealth groups benefit
themselves when the government provides help towards poor people.

To add, the distribution of ``helppoor'' scores were right skewed with
physical health. People with lower physical health agreed more that the
government should be helping poor people. This may because people who
are lower in health are more empathetic and therefore support the
altruistic attitude of helping poor people. This is contrary to my
hypothesis. Another interesting finding from the altruistic counts I
found was that wealthier people who are more satisfied financially have
a right skewed distribution in ``helpsick.'' Meaning they more
altruistic in the topic of the government helping sick people. Aligning
with my hypothesis.

Probing deeper into the relationships of these variables, I use
statistical models to quantify the extent of their relations. For the
categorical variables I use chi-square tests to determine whether the
altruistic responses were related to quality-of-life attributes. To
follow, of the pairs that were indeed dependant, I used ordinal logistic
regression to determine the extent of their relationships. These models
provided the most insight for this study. I found that financial/social
satisfaction along with mental health is significantly related to the
extent of altruistic responses. With increasing mental health and
satisfaction (social and financial) people agreed more that the
government should be helping people. Additionally, the extent that an
individual agrees that people in richer countries should be helping
people in poorer ones also increases. Interestingly, their view on other
people also becomes more optimistic. An individual who is more
financially/socially satisfied or has better mental health is more
likely to believe that other people will be helpful towards them. This
is in line with my hypothesis that people with a higher quality-of-life
are more likely to take altruistic stances.

Along with the study of categorical independent variables, I used linear
regression to study the relation of yearly income with the altruistic
attitudes. An interesting thing that I found was that with the increase
of income, the amount that an individual agrees that the government
should be helping black people financially (helpblk) specifically,
increases. This response could be non-reciprocal because the majority of
the respondents were likely not black (American demographics). This
means that people who are in a better financial state themselves can
take a more altruistic attitude towards others due to a freedom from
financial pressure.

To summarize these findings, I found that the psychological disposition
of an individual determined through their quality of life does indeed
have an effect on their altruistic attitude. While some of the findings
may be self serving (eg. poorer people agreeing with government help),
there are various instances where they are non-reciprocal in nature. The
general findings confirm my hypothesis; the better the quality of an
individual's life, the more inclined and able they are to take
altruistic attitudes. However, there are several weaknesses to the
study.

First and foremost, the breadth of this study for this paper in specific
is much too large. We have seen that there are plenty of variables that
can represent the quality of life of an individual. These include
objective values of their income, and subjective experiences/psychology.
Future studies should single out a psychological disposition such as
mental health and study its relationship with altruism. Even more
specifically, they may single out a factor that could be responsible for
mental health and analyse whether that also affects altruistic
attitudes.

To add, the survey questions identified may not exactly represent
non-reciprocal altruism specifically. For example, the idea of the
government helping people is relevant for everyone, including the
respondent and their direct relatives. Therefore, a survey specific to
non-reciprocal altruistic attitudes is required.

Lastly, hidden factors may be a big part why attributes such as
financial/social satisfaction and mental health are related to altruism.
The relation could also be reversed. People who are more altruistic may
be more satisfied with their life and have better mental health. Future
studies should study this reverse relationship, and attempt to identify
hidden factors that may affect both psychological stances.

\end{document}
